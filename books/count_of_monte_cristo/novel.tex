%Copyright © 2025 Brian Pomerantz. All Rights Reserved.

\documentclass[10pt,twoside]{memoir}

%------------------------------------------------
% Page layout for half-letter (5.5x8.5)
%------------------------------------------------
\setstocksize{8.5in}{5.5in}             % Final trimmed page size
\settrimmedsize{8.5in}{5.5in}{*}        % Trimmed size
\settypeblocksize{7in}{4.25in}{*}       % Text area (height x width)
\setbinding{3mm}                        % Additional gutter space for spine
\setlrmargins{*}{*}{1.2}                % Inner/outer margin ratio
\setulmargins{*}{*}{1.2}                % Top/bottom margin ratio
\setheaderspaces{*}{*}{1.0}
\checkandfixthelayout
\setlength{\parindent}{1em}
\setlength{\parskip}{0em}
\setlength{\voffset}{0.1in}

%------------------------------------------------
% Fonts
%------------------------------------------------
\usepackage{textcomp}
\usepackage{lmodern}
\usepackage[T1]{fontenc}
\usepackage{microtype}
\linespread{1.1}

%------------------------------------------------
% Formatting
%------------------------------------------------
\pagestyle{myheadings}

\aliaspagestyle{part}{empty}
\aliaspagestyle{chapter}{empty}

\newcommand{\mychapter}[1]{%
  \refstepcounter{chapter}%
  \markboth{\footnotesize THE COUNT OF MONTE CRISTO}{\footnotesize \MakeUppercase{#1}}
  \addcontentsline{toc}{chapter}{\makebox[2em][r]{\thechapter}\hspace*{1em}\MakeUppercase{#1}}%
  \begin{center}%
  \large \thechapter\par%
  %\vspace{0.5em}%
  \large \MakeUppercase{#1}\par%
  \end{center}
}

\newcommand{\mychapterstar}[1]{%
  \chapter*{\LARGE #1}%
  \markboth{\footnotesize \MakeUppercase{#1}}{\footnotesize \MakeUppercase{#1}}%
  \addcontentsline{toc}{chapter}{#1}
}

\addtocontents{toc}{\protect\thispagestyle{empty}}

\renewcommand{\thechapter}{\small \Roman{chapter}}

%\cftsetindents{chapter}{3em}{2.5em}

\setlength{\cftbeforepartskip}{1em}  % space before chapters
\setlength{\cftbeforechapterskip}{0.5em}  % space before chapters

%------------------------------------------------
% Front matter
%------------------------------------------------
\title{The Count of Monte Cristo}
\author{Alexandre Dumas \\ \\ \small \emph{Translated and with Introduction and Notes by} \\ \small Robin Buss}
\date{}

\begin{document}
\frontmatter

\thispagestyle{empty}
\begin{center}
{\small THE COUNT OF MONTE CRISTO}
\end{center}
{\small
Alexandre Dumas was born in 1802 at Villers-Cotter\^ets. His father, the
illegitimate son of a marquis, was a general in the Revolutionary
armies, but died when Dumas was only four. He was brought up in
straitened circumstances and received very little education. He joined
the household of the future king, Louis-Philippe, and began reading
voraciously. Later he entered the \emph{c\'enacle} of Charles Nodier and
started writing. In 1829 the production of his play, \emph{Henri III et
sa cour}, heralded twenty years of successful playwriting. In 1839 he
turned his attention to writing historical novels, often using
collaborators such as Auguste Maquet to suggest plots or historical
background. His most successful novels are \emph{The Count of Monte
Cristo}, which appeared during 1844--5, and \emph{The Three Musketeers},
published in 1844. Other novels deal with the wars of religion and the
Revolution. Dumas wrote many of these for the newspapers, often in daily
instalments, marshalling his formidable energies to produce ever more in
order to pay off his debts. In addition, he wrote travel books,
children's stories and his \emph{M\'emoires} which describe most amusingly
his early life, his entry into Parisian literary circles and the 1830
Revolution. He died in 1870.

\bigskip

\noindent Robin Buss is a writer and translator who contributes regularly to
\emph{The Times Educational Supplement}, \emph{The Times Literary
Supplement} and other papers. He studied at the University of Paris,
where he took a degree and a doctorate in French literature. He is
part-author of the article `French Literature' in \emph{Encyclopaedia
Britannica} and has published critical studies of works by Vigny and
Cocteau, and three books on European cinema, \emph{The French through
Their Films} (1988), \emph{Italian Films} (1989) and \emph{French Film
Noir} (1994). He is also part-author of a biography, in French, of King
Edward VII (with Jean-Pierre Navailles, published by Payot, Paris,
1999). He has translated a number of other volumes for Penguin,
including Jean Paul Sartre's \emph{Modern Times}, Zola's
\emph{L'Assommoir} and \emph{Au Bonheur des Dames}, and Albert Camus's
\emph{The Plague}.
}

\cleardoublepage

\maketitle
\thispagestyle{empty}
\clearpage

% Copyright page
\thispagestyle{empty}
\vspace*{\fill}
\begin{center}
\tiny
First published 1844--5\\
This translation first published 1996\\
Reissued with new Chronology and Further Reading 2003
\vspace{1\baselineskip}

Copyright \textcopyright\ Robin Buss, 1996, 2003\\
All rights reserved\\
\vspace{1\baselineskip}

The moral right of the translator has been asserted\\
\vspace{1\baselineskip}

This edition published by Brian Pomerantz, 2025\\
\vspace{1\baselineskip}

Printed in the United States of America.
\end{center}
\vspace*{\fill}
\clearpage


% Table of contents
\tableofcontents*
\cleardoublepage
%\phantom{foo}
%\thispagestyle{empty}
%\cleardoublepage

\mychapterstar{Chronology}

\begin{description}
\item[\textbf{1802}] Alexandre Dumas is born at Villers-Cotter\^ets, the third
child of Thomas-Alexandre Dumas. His father, himself the illegitimate
son of a marquis and a slave girl of San Domingo, Marie-Cessette Dumas,
had had a remarkable career as a general in the Republican, then in the
Napoleonic Army.

\item[\textbf{1806}] General Dumas dies. Alexandre and his mother, Elisabeth
Labouret, are left virtually penniless.

\item[\textbf{1822}] Dumas takes a post as a clerk, then in 1823 is granted a
sinecure on the staff of the Duke of Orl\'eans. He meets the actor Talma
and starts to mix in artistic and literary circles, writing sketches for
the popular theatre.

\item[\textbf{1824}] Dumas' son, Alexandre, future author of \emph{La Dame aux
cam\'elias}, is born as the result of an affair with a seamstress,
Catherine Lebay.

\item[\textbf{1829}] Dumas' historical drama, \emph{Henri III et sa cour}, is
produced at the Com\'edie-Fran\c{c}aise. It is an immediate success, marking
Dumas out as a leading figure in the Romantic movement.

\item[\textbf{1830}] Victor Hugo's drama \emph{Hernani} becomes the focus of
the struggle between the Romantics and the traditionalists in
literature. In July, the Bourbon monarchy is overthrown and replaced by
a new regime under the Orl\'eanist King Louis-Philippe. Dumas actively
supports the insurrection.

\item[\textbf{1831}] Dumas' melodrama \emph{Antony}, with its archetypal
Romantic hero, triumphs at the Th\'e\^atre de la Porte Saint-Martin.

\item[\textbf{1832}] Dumas makes a journey to Switzerland which will form the
basis of his first travel book, published the following year.

\item[\textbf{1835}] Dumas travels to Naples with Ida Ferrier (whom he will
later marry), has a passionate affair in Naples with Caroline Ungher and
falls in love with Italy and the Mediterranean.

\item[\textbf{1836}] Triumph of Dumas' play \emph{Kean}, based on the
personality of the English actor whom Dumas had seen performing in
Shakespeare in 1828.

\item[\textbf{1839}] \emph{Mademoiselle de Belle-Isle}. Dumas' greatest success
in the theatre.

\item[\textbf{1840}] Dumas marries Ida Ferrier. He travels down the Rhine with
G\'erard de Nerval and they collaborate on the drama \emph{L\'eo Burckart}.
Nerval introduces Dumas to Auguste Maquet who will become his
collaborator on many of his subsequent works.

\item[\textbf{1841}] Spends a year in Florence.

\item[\textbf{1844}] The year of Dumas' two greatest novels: \emph{The Three
Musketeers} starts to appear in serial form in March and the first
episodes of \emph{The Count of Monte Cristo} follow in August. Dumas
starts to build his Ch\^ateau de Monte-Cristo at St-Germain-en-Laye. He
separates from Ida Ferrier.

\item[\textbf{1845}] \emph{Twenty Years After}, the first sequel to \emph{The
Three Musketeers}, appears at the beginning of the year. In February,
Dumas wins a libel action against the author of a book accusing him of
plagiarism. Publishes \emph{La Reine Margot}.

\item[\textbf{1846}] Dumas travels in Spain and North Africa. Publishes
\emph{La Dame de Monsoreau}, \emph{Les Deux Diane} and \emph{Joseph
Balsamo}.

\item[\textbf{1847}] Dumas' theatre, the Th\'e\^atre Historique, opens. It will
show several adaptations of his novels, including \emph{The Three
Musketeers} and \emph{La Reine Margot}. Serialization of \emph{The
Vicomte de Bragelonne}, the final episode of the \emph{The Three
Musketeers}.

\item[\textbf{1848}] A revolution in February deposes Louis-Philippe and brings
in the Second Republic. Dumas stands unsuccessfully for Parliament and
supports Louis-Napol\'eon, nephew of Napoleon I, who becomes President of
the Republic.

\item[\textbf{1849}] Dumas publishes \emph{The Queen's Necklace}.

\item[\textbf{1850}] Dumas is declared bankrupt and has to sell the Ch\^ateau de
Monte-Cristo and the Th\'e\^atre Historique. Publishes \emph{The Black
Tulip}.

\item[\textbf{1851}] In December, Louis-Napol\'eon seizes power in a coup d'\'etat,
effectively abolishing the Republic. A year later, the Second Empire
will be proclaimed. Victor Hugo goes into exile in Belgium where Dumas,
partly to escape his creditors, joins him.

\item[\textbf{1852}] Publishes his memoirs.

\item[\textbf{1853}] In November, returns to Paris and founds a newspaper,
\emph{Le Mousquetaire}. Publishes \emph{Ange Pitou}.

\item[\textbf{1858}] Founds the literary weekly, \emph{Le Monte-Cristo}. Sets
out on a nine-month journey to Russia.

\item[\textbf{1860}] Meets Garibaldi and actively supports the Italian struggle
against Austria. Founds \emph{L'Independente}, a periodical in Italian
and French. Garibaldi is godfather to Dumas' daughter by Emilie Cordier.

\item[\textbf{1861--1870}] Dumas continues to travel throughout Europe and to
write, though his output is somewhat reduced. None the less in the final
decade of his life, he published some six plays, thirteen novels,
several shorter fictions, a historical work on the Bourbons in Naples
and a good deal of journalism. He had a last love affair, with an
American, Adah Menken, and indulged his lifelong passions for drama,
travel and cookery.

\item[\textbf{1870}] Alexandre Dumas dies on 5 December in Dieppe.
\end{description}

\mychapterstar{Introduction}

`Ah, a children's novel,' a Russian film-maker remarked when I told her
that I was translating \emph{The Count of Monte Cristo}. The comment was
not intended to be disparaging, merely descriptive; and many people, in
different cultures, would tend to agree with the categorization. Most
will derive their idea of the novel, not from having read it, but
because a kind of abstract of the storyline exists as part of the common
culture: innocent man imprisoned, meets fellow-prisoner who directs him
to a buried fortune, escapes and plots revenge. It has been adapted for
film, television and the theatre, as well as being translated, abridged
and imitated in print. It has supplied material for cartoons and comedy:
the Irish comedian Dave Allen used to do a series of sketches around the
theme of a young man (Dant\`es) breaking through a dungeon wall and
encountering an old, bearded prisoner (Abb\'e Faria). Some events in the
story are so well-known that they exist apart from the novel, like
Robinson Crusoe's discovery of Man Friday's footprint, or incidents and
characters from \emph{Treasure Island} and \emph{Frankenstein}.
\emph{The Count of Monte Cristo} is one of the great popular novels of
all time and, like other popular novels, it has suffered the fate of
being treated as not fully `adult' fiction; like children's fiction, it
seems to inhabit a realm outside its creator's biography and the period
when it was written.

On the other hand, there are not many children's books, even in our own
time, that involve a female serial poisoner, two cases of infanticide, a
stabbing and three suicides; an extended scene of torture and execution;
drug-induced sexual fantasies, illegitimacy, transvestism and
lesbianism; a display of the author's classical learning, and his
knowledge of modern European history, the customs and diet of the
Italians, the effects of hashish, and so on; the length would, in any
case, immediately disqualify it from inclusion in any modern series of
books for children. Most important of all, perhaps, is the fact that the
author himself never thought of this as `a children's novel'. Yet
already in the earliest translations into English, with their omission
or subtle alteration of material that might be considered indelicate by
Victorian readers, and of some passages (for example, references to
classical literature) that might be thought to hold up the story, one
can see the start of a process of transformation, from `novel' to `genre
novel' -- which means, ultimately, almost any kind of genre novel:
`adventure', `romance', `thriller' and, if you like, `children's novel'.
This is the usual fate of books that fail to meet the criteria for
serious, `literary' fiction.

Dumas himself must bear some of the responsibility. During his most
productive decade, from 1841 to 1850, he wrote forty-one novels,
twenty-three plays, seven historical works and half a dozen travel
books. The nineteenth century was an age of mass production, which is
precisely why Art felt the need to distinguish itself by its
individuality and craftsmanship: `Alexandre Dumas and Co., novel
factory', was the contemptuous title given to one critical pamphlet,
published at the same time as this novel, in 1845. Moreover it was known
that Dumas wrote for money, at so much a line, and that he used at least
one collaborator, Auguste Maquet, who would make chapter outlines for
him and do research. There was a vast difference between this industrial
labour and the monastic devotion to the cause of art that kept Gustave
Flaubert at his desk for seven hours a day as he wrote and rewrote
\emph{Madame Bovary} (1857). In the history of the novel, Dumas and
Flaubert stand near the head of divergent streams.

Alexandre Dumas was born on 24 July 1802; or, rather, since the
Republican Calendar was still in force, on 5th Thermidor, Year x, in the
little town of Villers-Cotter\^ets, near Soissons. His father was a
general in the revolutionary armies, himself the illegitimate son of a
marquis, Antoine-Alexandre Davy de la Paillerie, and a black slave from
the island of Santo Domingo, Marie Dumas. In 1806, General Dumas died,
leaving his family virtually without resources. The child had little
education, enough however to allow him to read \emph{Robinson Crusoe}
and \emph{The Arabian Nights}, and to cultivate his handwriting. In
1823, thanks to the second of these, he found employment in Paris,
copying documents for Louis-Philippe, Duke of Orl\'eans.

The 1820s were a marvellous time for an aspiring young writer in Paris.
The two rival literary ideologies, of Classicism and Romanticism, were
engaged in a mock-heroic combat for the soul of French literature.
Classicism stood for universal themes, refinement, purity of language,
clear division of literary genres and (despite its debt to the
literature of the classical world) the peculiarly French ethos of the
dramatist Racine. Romanticism meant energy, modern subject-matter,
mixing genres and openness to foreign influences, particularly that of
Shakespeare, the Romantic dramatist \emph{par excellence}. It was in the
theatre that the confrontation would chiefly take place.

Racine had based his plays on stories from classical Greece or on
biblical history, both of which offered `universal' events and
characters. Shakespeare, like the German playwright Schiller, had dealt
with subjects from modern history, which were national and particular
rather than universal. In France, especially, the period that followed
the great upheavals of the Revolution, the Empire and the Restoration
was one which had an urgent need to make sense of the past.
Shakespeare's history plays -- and, still more, the historical novels of
Walter Scott -- were models of how this could be done, drawing on the
imagination as well as on scholarship. In 1828, Dumas, who had already
tried his hand at a couple of plays and some short stories, submitted a
historical drama to the Com\'edie Fran\c{c}aise entitled \emph{Henri III et sa
cour}. It was a typically Romantic work, ignoring the `unities' of time,
place and action, and written in prose, rather than the conventional
medium of verse. It underwent the usual ritual of a public reading and,
at its first night on 10 February 1829, scored a triumphant success and
was warmly applauded by the author's employer, Louis-Philippe. In the
following year, Louis-Philippe became king, after a liberal revolution
that was supposed to bring in a constitutional monarchy. Dumas welcomed
it; so did the former ultra-monarchist, Victor Hugo.

During the next twenty years, Dumas was (with Hugo and Alfred de Vigny)
the leading dramatist of the new movement -- and, of the three, easily
the most prolific. Perhaps too much so: overnight, after the first
performance of \emph{Christine} in 1830, while Dumas was asleep, Hugo
and Vigny rewrote the play, reducing it to a more manageable size.
Despite this, Dumas' play \emph{Antony} (1831) is an essential work of
the Romantic period, as representative as Hugo's \emph{Hernani} or
Vigny's \emph{Chatterton}, and more successful with its audiences than
either. But the theatre is the very opposite of a monastic cell or an
ivory tower. Collaboration is not only the norm, but inevitable,
feedback from the public is instantaneous, work has to be produced to
satisfy demand, and there is an immediate relationship between the
author's output and what comes in through the box office. In the
theatre, Dumas learned the rudiments of literary production.

On one occasion, Charles-Jean Harel, director of the Od\'eon theatre, is
supposed to have locked Dumas into a room, away from his mistress, for a
week, until he had completed the manuscript of \emph{Napol\'eon} (1831).
The huge growth in the periodical press during the 1820s saw the
invention of the \emph{feuilleton} -- not in the sense of a regular
column by one writer, but of a novel published in instalments; Dumas
claimed to have invented the \emph{roman feuilleton} with \emph{La
Comtesse de Salisbury}, published in \emph{La Presse} in 1836. By the
early 1840s he was writing more novels than plays, mainly (but by no
means exclusively) historical fiction which, as I have already
mentioned, was one of the most popular genres; it was also taken
seriously as a means of exploring the past. He did, incidentally, write
a book for children at this time: \emph{Le Capitaine Pamphile} (1839).

Travel, to which he was addicted, helped to stave off boredom, providing
the material for travel books, while translation filled in the remaining
gaps in the working day. Like Balzac, he was a man of huge appetites:
food, sex, work, sleep, pleasure, leisure, movement, excitement. In
Italy, he found love, opera, colour and the Mediterranean: he visited
Naples and Palermo in 1835, stayed a year in Florence in 1841 and
returned in 1843 for a visit that included Sicily. The following year
saw the publication of his first great historical novel, \emph{Les Trois
Mousquetaires/The Three Musketeers}, and on 28 August 1844 \emph{Le
Journal des D\'ebats} began publication of \emph{The Count of Monte
Cristo}. It was an immediate success, translated, adapted, pirated
\ldots{} in short, a popular novel.

\bigskip

\noindent It was also, very clearly, a work of its time. The plot was inspired by
the true-life story of Fran\c{c}ois Picaud, which Dumas found in Jacques
Peuchet's \emph{Police d\'evoil\'ee: M\'emoires historiques tir\'es des archives
de Paris} \ldots{} (1838), a collection of anecdotes from the Parisian
police archives.\textsuperscript{1}

Briefly, the story is this: Picaud, a young man from the south of
France, was imprisoned in 1807, having been denounced by a group of
friends as an English spy, shortly after he had become engaged to a
young woman called Marguerite. The denunciation was inspired by a caf\'e
owner, Mathieu Loupian, who was jealous of Picaud's relationship with
Marguerite.

Picaud was eventually moved to a form of house-arrest in Piedmont and
shut up in the castle of Fenestrelle, where he acted as servant to a
rich Italian cleric. When the man died, abandoned by his family, he left
his money to Picaud, whom he had come to treat as a son, also informing
him of the whereabouts of a hidden treasure. With the fall of Napoleon
in 1814, Picaud, now called Joseph Lucher, was released; in the
following year, after collecting the hidden treasure, he returned to
Paris.

Here he discovered that Marguerite had married Loupian. Disguising
himself, and offering a valuable diamond to Allut, the one man in the
group who had been unwilling to collaborate in the denunciation, he
learned the identity of his enemies. He then set about eliminating them,
stabbing the first with a dagger on which were printed the words:
`Number One', and burning down Loupian's caf\'e. He managed to find
employment in Loupian's house, disguised as a servant called Prosper.
However, while this was going on, Allut had fallen out with the merchant
to whom he had resold the diamond, had murdered him and had been
imprisoned. On coming out of jail, he started to blackmail Picaud.
Picaud poisoned another of the conspirators, lured Loupian's son into
crime and his daughter into prostitution, then finally stabbed Loupian
himself. But he quarrelled with Allut over the blackmail payments and
Allut killed him, confessing the whole story on his deathbed in 1828.

It is obvious both how directly Dumas was inspired by Peuchet's account
of this extraordinary tale, and how radically he transformed it;
incidentally, he used another chapter of Peuchet's book as the basis for
the story of Mme de Villefort. One important step in the transformation
from `true crime' to fiction was to shift the opening of the tale from
Paris to Marseille, giving the novel its Mediterranean dimension. Though
most of the action still takes place in Paris (apart from a few
excursions elsewhere, all the novel between Chapters {\small XXXIX} and {\small CIV} is
set in Paris), the sea is always present as a figure for escape and
freedom, while the novel uses the southern origins of its characters as
a means to evoke that exotic world of the Mediterranean littoral that
had so fascinated French writers and artists since the 1820s. The
Mediterranean is the point where the cultures of Europe meet those of
the Orient, and the region had been in the forefront of people's minds
since the 1820s, because of the Greek struggle for independence and the
French conquest of Algeria.

Both of these are directly present in the novel: one of its young
characters is a soldier who has just returned from Algeria, another sets
off to fight in the colony. As for Greece, which rebelled against the
Turks in the 1820s, it inspired much fervour among European Romantics,
most famously Lord Byron. The story of Ali (1741--1822), Pasha of Janina
(Jannina) in Albania, plays a direct part in the novel and also takes us
into the Oriental world that fascinated the French Romantics. `The
Orient,' Victor Hugo wrote in the preface to his early collection of
poems, \emph{Les Orientales} (1829), `both as an image and as an idea,
has become a sort of general preoccupation for people's minds as much as
for their imaginations, to which the author has perhaps unwittingly
succumbed. As if of their own accord, Oriental colours have come to
stamp their mark on all his thoughts and reveries \ldots' -- as they
also marked the paintings of Ingres and Delacroix. When we meet Hayd\'ee
in Chapter {\small XLIX}, she is lying on a heap of
cushions, wearing her native Albanian costume, smoking a hookah and
framed in a doorway, `like a charming painting'.

Italy was another Mediterranean land that held a powerful appeal for the
Romantics, and in particular for Dumas. All the components of this
appeal are in the novel: the classical world (the night visit to the
Colosseum), the excitement of travel (Chapter {\small XXXIII}, `Roman Bandits'),
the cruel justice of the Papal states (Chapter {\small XXXV}, `La Mazzolata'), colourful
spectacle (Chapter {\small XXXVI}, `The Carnival in Rome'), the Christian
past (Chapter {\small XXXVII}, `The
Catacombs of Saint Sebastian'). The story of Luigi Vampa could have come
directly from one of Stendhal's \emph{Italian Chronicles}, the
description of the Colosseum at night from one of Byron's or Shelley's
letters. There is also a good deal of wit -- and the fruit of personal
experience -- in Dumas' portrayal of the modern Romans and the
day-by-day experience of the Grand Tour. Like all the most skilled
popular writers, he offers his readers a mixture of the unfamiliar and
the expected: references to places, people and events that will conjure
up a whole complex of images and ideas -- we have here the notion of
Italy as it was perceived in France in the 1840s, through literature and
art -- combined with those intimate touches that allow readers to
experience the sensations of being there. Reading Dumas, we know how it
felt to be swept up in the crowd at the Carnival, to travel in a
carriage through the Roman streets, to stay in a \emph{pensione}. We can
easily recognize the proud bandit, the bustling hotelier, the alluring
woman in the Carnival crowd.

All these are described with as much economy as possible in order to
avoid holding up the narrative. This is one reason why the popular novel
tends to reinforce rather than to challenge prejudices -- although, in
one case, Dumas' novel reversed a prejudice, namely that Marseille was,
in the words of Murray's \emph{Handbook for Travellers in France}
(1847), `a busy and flourishing city \ldots{} {[}but one that{]} has few
fine public buildings or sights for strangers'. \emph{The Count of Monte
Cristo}, on the contrary, with its intimate topography of the area
around the old port and its dramatization of Marseille as the focus of
mercantile activity, the meeting-place of Mediterranean cultures and the
gateway to the Arab Maghreb, is a good deal more flattering than
Murray's \emph{Handbook}. Dumas was allegedly thanked by a Marseillais
cab-driver for promoting the city.

Apart from this novel depiction of France's major sea-port, however,
Dumas offers his readers a Rome, and an Orient, that are very much what
they would expect: the first colourful, tuneful, proud and cruel, the
second decadent and opulent. But he adds those little details that
compel belief in what he is describing: the precise information about
Carnival etiquette, the street-by-street itinerary of a drive round the
walls of Rome, the horrifying description of a Roman execution, sketches
of character or scenery that he has culled from his own memories of
staying at Signor Pastrini's hotel when he visited Rome in 1835. His
passages on sailing ships spare us none of the technicalities of sails
and masts; his descriptions of the effects of opium convince us that he
had experienced them. And, in much the same way, he adds touches of
erudition: a quotation from Horace, a reference to \emph{Hamlet} -- all
of which are meant to reassure us that we are in reliable hands. At
times he even allows himself the luxury of a longer purple passage
(perhaps a sunset over the Mediterranean) to show that he can do that,
too.

All this helps to justify his claim that he has transformed Peuchet's
material into something infinitely more valuable. Peuchet's account of
the Picaud case, he wrote, was `simply ridiculous \ldots{} {[}but{]}
inside this oyster, there was a pearl. A rough, shapeless pearl, of no
value, waiting for its jeweller.' And, of course, the essential
transformation that the jeweller makes to Peuchet's story lies in the
character of the Count.

To begin with, we have Edmond Dant\`es, a man who could well be first
cousin to the shoemaker, Fran\c{c}ois Picaud. Betrayed by a jealous rival
and an ambitious colleague, sent to the fortress prison of If by a
magistrate who cannot afford to let the facts come out, Dant\`es goes
through a kind of burial and resurrection. Educated by Abb\'e Faria and
possessor of a limitless fortune, he can re-emerge into the world, not
as the cobbler Picaud, content to stab or poison those responsible for
his misfortune, but as an instrument of divine justice. Dumas' first,
vital departure from Peuchet is to make Monte Cristo only indirectly the
avenger: his `victims' are all, in reality, destroyed by their own past
misdeeds which Monte Cristo uncovers.

As the man who brings the truth to light and uses the discovery to
punish the wrongdoer, Monte Cristo is the forerunner of the detective,
that central figure in modern popular fiction. In fact, there is more
than one reference in the novel to deductive methods that resemble those
pioneered by Edgar Allan Poe (`A Manuscript Found in a Bottle', `The
Gold Bug', `The Murders in the Rue Morgue') -- for example, in the way
in which Abb\'e Faria deciphers the will showing where the treasure was
hidden, Dant\`es' own analysis of where exactly it is concealed on the
island and, earlier, Faria's explanation of Dant\`es' imprisonment. Note
that, like the intellectual exercises of Monte Cristo's opium-taking
successor later in the century, Sherlock
Holmes,\textsuperscript{2} these deductions at
first amaze those who have not been able to follow the logic behind them
or who do not have the expertise to know, for example, when something
has been written with the left hand by a right-handed person. But, of
course, Faria is not really a Holmesian detective: the stereotype in
Dumas' mind is that of the eighteenth-century \emph{philosophe}, a
believer in the power of reason and a student of human nature. What
Faria lacks (ironically, since everyone around thinks him mad, insanely
obsessed with his fictitious treasure) is the Holmesian neuroses: the
brooding violin and the opium stupor. These come from a different
fictional archetype.

So does Monte Cristo, even though he is not averse at times to applying
Faria's deductive logic (and shares Holmes's talent for disguise).
Having emerged in 1829 from his entombment, found his treasure,
discovered the fate of his father and Merc\'ed\`es, and repaid his debt to
Morrel, Dant\`es then disappears for another nine years, about which the
reader is told virtually nothing. This second period of latency is not
strictly a remaking but an effacement: the character who re-emerges in
the novel as the Count of Monte Cristo is shrouded in mystery; we only
assume, at first, that he is identical to Edmond Dant\`es on the slender
evidence of their using the same pseudonym: `Sinbad the Sailor'. He is a
dark, brooding figure, pale-faced, with an aversion to food and
apparently devoid of some human feelings: he takes an evil delight in
terrifying his young friends, Albert and Franz, with the spectacle of an
execution. He is also, as they learn later, on good terms with the
bandit, Luigi Vampa.

The appearance of this deathly-white apparition in a box at the Roman
opera immediately evokes two other personalities who played a major role
in popular mythology in France in the Romantic era. The first is Lord
Byron, a real-life character who very early was confused with his
fictional creations, Childe Harold, Manfred and Don Juan -- all the more
so in France, where the poetry might be known only in translation. The
image was that of a young but world-weary hero, tormented by nameless
despairs. The second figure was that of the vampire, associated with the
first through the story \emph{Lord Ruthwen, or The Vampire}, which was
attributed to Byron (though in fact written by his companion, Polidori).
This was not by any means the only vampire to be found in France at the
time: the theatre, notably during the 1820s, was haunted by the Undead:
English vampires, comic vampires, female vampires \ldots{} The nature of
the vampire was perhaps not so precisely codified as it was to be later,
especially by Bram Stoker in that tale of another mysterious count:
garlic, stakes, crosses, Transylvania, the vampire's native soil in the
coffin which he keeps in the basement, these are not yet firmly
established in the mythology. But the figure is there, and elements of
the legend are specifically ascribed to the Byronic figure of Monte
Cristo.

What I would like to suggest is that Dumas' novel stands at a crucial
point in the development of modern popular fiction, drawing into the
genre elements from Romantic literature, popular theatre, history and
actuality, and wrapping them up in a narrative carefully enough
constructed and dramatic enough to hold the attention of a growing
reading public with a great appetite for fiction. They would satisfy it
not only with books, but also with the newspaper serials which had
brought fame and fortune to Dumas' precursor in the genre, Eug\`ene Sue.

\emph{Monte Cristo} owes its existence directly to Sue's \emph{Myst\`eres
de Paris} (1842--3): it was precisely the success of Sue's tales that
made Dumas' publishers demand a novel, rather than the historical guide
to Paris that they had originally commissioned. Sue's appeal to the
public was the ability to suggest the existence of a sinister underworld
of crime and intrigue behind the fa\c{c}ade of a Paris that was familiar to
most of his readers. The growth of the nineteenth-century metropolis led
to a whole literature of the urban life, later exploited on film, in
which the city is no longer seen as a place of civilized, `urbane'
living and safety from attack, but as a menacing sub-world, in which
human beings prey on one another or suffer fearful bouts of loneliness,
alienation and ennui. A machine devised to supply every need of
civilized humanity in one place has become a monster enclosing every
form of vice and depravity. Only in England did murder continue to take
place in country houses.

As noted earlier, Paris is the setting for the greater part of the book,
but the episodes in Marseille and Rome enrich it enormously. We do have,
at the very centre, a very Parisian murder story, joined to a rather
trite Parisian romance, and Dumas locates every event precisely on the
city map, so that all the addresses are real; but the overall impression
left by the novel is of something far larger in scope than a tale of
Parisian wrongdoing and revenge. The episodes in Marseille and Rome may
have been added after the book was begun -- it was Dumas' collaborator,
Maquet, who suggested actually recounting Dant\`es' arrest and
imprisonment, instead of starting the novel in Rome and then
transferring the action rapidly to Paris; yet the first section proves
absolutely crucial. Where the count, in himself, descends at times to
the level of a melodramatic stage avenger, Dant\`es is a compelling
character, and it is the figure of Edmond Dant\`es (whom we feel obscurely
present in his later incarnation) which gives the latter depth and
weight.

The re-emergence of the other characters after the latency period of
Dant\`es' imprisonment is more of a problem. Caderousse is essentially
unchanged, Danglars more or less unrecognizable. Fernand offers the
least plausible transformation of all, from the brave and honest
Spaniard with a sharp sense of honour, whom we meet in the early
chapters, to the Parisian aristocrat whose life seems to have been
dedicated to a series of betrayals. Fernand/Morcerf seems to confirm a
criticism of Dumas and of popular novels in general, namely that they
tend to sacrifice character to plot.

In some respects, though, in Dumas' novel the reverse is true: Dumas'
novel is dictated by character. But it is character viewed more as an
imaginative construct than as a psychological novelist would conceive
it. The count himself is a poetic character, a creature of the
imagination who draws on elements from myth as much as from everyday
psychological observation. And, while Madame de Villefort, Valentine,
Morrel and some others in Dumas' huge cast may be `flat' characters,
performing a largely functional role in the development of the story,
there are several secondary figures to whom this does not apply, notably
Eug\'enie Danglars and Albert de Morcerf. In many ways, Eug\'enie is
Valentine's twin. Both women are heiresses to large fortunes, both are
presented with the alternatives of subjecting themselves to their
father's will and marrying men whom they do not love or being confined
to a convent. But where Valentine is willing to submit, Eug\'enie is not.
Her lesbianism may be a trait of personality, but it is also an
expression of her desire for independence.

There is far more to \emph{The Count of Monte Cristo} than merely a tale
of adventure and revenge. None the less, it is a book that many people
first encounter and enjoy during their teens. Not long after Dumas'
death, Victor Hugo wrote a letter to his friend's son, Alexandre Dumas
\emph{fils}, in which he praised Dumas as a writer of universal appeal
and added `He creates a thirst for reading.' After more than 150 years,
\emph{The Count of Monte Cristo} remains one of the most popular and
widely read novels in world literature; its longevity singles it out as
almost unique among `popular' novels. For many of its readers, despite
its length, it seems all too short; we want to spend more time with the
count and the other characters in the book, more time in its bustling
world of drama and passion. Creating that thirst for more is among
Dumas' great contributions to literature.

\vspace*{\fill}

\begin{center}
\large Notes
\end{center}

\bigskip

\noindent {\small \textbf{1.} }Peuchet's text is reprinted in the edition of
the novel by Claude Schopp (Robert Laffont, Paris, 1993).

\noindent {\small \textbf{2.} }The link with Conan Doyle is actually
strengthened by the more obvious similarities in the field of historical
fiction (for example, between Doyle's \emph{The White Company} and
Dumas' \emph{The Three Musketeers}). Conan Doyle may have consciously
followed Dumas in his historical novels and unconsciously in creating
Holmes.

\clearpage

\begin{center}
\Large Further Reading
\end{center}

\bigskip
\bigskip

\hangindent=2em
\noindent Hemmings, F. W. J., \emph{The King of Romance}, Hamish Hamilton, London,
1979.

\hangindent=2em
\noindent Maurois, Andr\'e, \emph{Three Musketeers}. \emph{A Study of the Dumas
Family}, translated by Gerard Hopkins, Jonathan Cape, London, 1957.

\hangindent=2em
\noindent Schopp, Claude, \emph{Alexandre Dumas}. \emph{Genius of Life},
translated by A. J. Koch, Franklin Watts, New York, 1988.

\hangindent=2em
\noindent Stowe, Richard, \emph{Dumas}, Twayne Publishers, Boston, 1976.

\cleardoublepage

\mychapterstar{A Note on the Text}

\emph{The Count of Monte Cristo} began publication in parts, in the
\emph{Journal des D\'ebats}, in August 1844; this continued until January
1846, by which time the first, 18-volume edition had been published by
P\'etion (1845--6). The second, third and fourth editions appeared in
1846, and there were several pirate editions in the 1840s. The book
continued to be re-published throughout the century, the last edition in
Dumas' lifetime being that published by Michel L\'evy in 1865.

The novel was rapidly translated into English (in England, in
\emph{Ainsworth's Magazine}, 1845, and by Emma Hardy, 1846; and in
America in 1846); and into: Danish (1845--6); Swedish (1846); Italian
(by Oreste Ferrario, 1847); Spanish (1858); Norwegian (1881--2); and
German (1902). The first stage adaptation was the one made by Dumas and
Maquet themselves (in 1848, in two parts; in 1851, in two parts; finally
shown as one single performance, in five acts and 12 tableaux, in 1862).
Long before this, however, there had been a stage parody by Deforges and
Claireville, \emph{Le Comte de Monte-Fiasco} (1847) -- a further tribute
to the notoriety of the work.

The most recent stage version was a new adaptation performed in England
in 1994. There have been condensed editions, children's editions and a
comic-book version. There were film adaptations in 1908 (USA), 1913
(USA), 1914 (France), 1934 (USA), 1942 (France), 1953 (France), 1961
(France), 1975 (USA) and 2001 (USA), as well as television versions. The
strength of the story is enough to explain why the novel has proved so
adaptable to other media, despite its length: the central themes of
betrayal, wrongful imprisonment and revenge are clear enough to allow
many of the sub-plots to be discarded for reasons of time or space.

Inevitably, something will be lost: there is simply so much there; and,
from the earliest days, the process undergone by Dumas' novel was one of
reduction, as if the original was too vast to stand by itself. There is
also the matter of the historical moment at which \emph{The Count of
Monte Cristo} appeared.

The mid-nineteenth century saw a continuing struggle to establish the
credentials of the literary novel, by giving it the dual aims that
Stendhal had helped to pioneer, which were those of exploring the
enduring features of human psychology and analysing a particular state
of human society. In contrast to such enterprises, fiction which
involved larger-than-life characters and implausible situations, Gothic
horrors, melodramatic incidents and so on appeared mere entertainment.
The gradual emergence of realism in the European novel was not
altogether to the advantage of Dumas, whose image was less that of the
austere priest than the jolly friar, and whose novels poured out of a
factory, the purpose of which was to create entertainment and sell it
for money.

This explains why, though Thackeray admitted finding the book impossible
to put down, English novelists like George Eliot considered that `the
French' -- Dumas, Hugo and Balzac -- were mistakenly tempted to deal
with the exception rather than the rule: to look for melodramatic
situations and characters, when they should be exploring the everyday
life that revealed what is enduring in human nature. It is not hard,
anyway, to guess that the author of \emph{Middlemarch} and \emph{The
Mill on the Floss} would not find much to please her in \emph{The Three
Musketeers} or \emph{The Count of Monte Cristo}.

There is also the question of Dumas' style, which is usually
unremarkable; and the fact that he wrote his great novels in
collaboration with Maquet, which does not accord with the idea of the
author as sole creator. No wonder people have thought they could treat
\emph{Monte Cristo} as a treasure-trove rather than a sacred text, or
that the many adaptations, abbreviations and reworkings of it have been
done with a good deal less reverence (and consequently, more often than
not, a good deal more success) than, say, Claude Chabrol brought to his
film version of \emph{Madame Bovary}. In the main, its fate has been
that of most nineteenth-century `adventure' novels: it has been treated
as mere entertainment for adults or literature for the young.

The truth is that, more because of the subject-matter than because of
its length, the novel has had to be tampered with before it can be
offered to young readers; or, as one may conjecture, to readers in
mid-Victorian England. And, because this is merely a `popular' novel, as
well as one which represents a huge amount of work for a translator,
there has been little enthusiasm in the English-speaking world for
re-translating it.

Claude Schopp's edition (Robert Laffont, 1993), which lists the main
foreign translations, records nothing into English since 1910. The most
readily available edition in Britain at the moment reproduces the
anonymous translation first published by Chapman \& Hall in 1846. Its
editor for the Oxford World's Classics series (1990), David Coward,
writes that `with one or two exceptions, the small number of ``new''
translations since made have drawn heavily upon \ldots{} this classic
version'.

Anyone who has read \emph{The Count of Monte Cristo} only in this
`classic version' has never read Dumas' novel. For a start, the
translation is occasionally inaccurate and is written in a
nineteenth-century English that now sounds far more antiquated than the
French of the original does to a modern French reader: to mention one
small point in this connection, Dumas uses a good deal of dialogue (he
wrote by the line), and the constant inversions of `said he' and `cried
he' are both irritating and antiquated. There are some real oddities,
like the attempt to convey popular speech (which does not correspond to
anything in Dumas), when the sailor in Chapter {\small XXV}
says: `that's one of them nabob gentlemen from Ingy
[\emph{sic}], no doubt \ldots' Even aside from that, most of the
dialogues in this nineteenth-century translation, in which the
characters utter sentences like: `I will join you ere long', `I confess
he asked me none' and `When will all this cease?', have the authentic
creak of the Victorian stage boards and the gaslit melodrama.

It can be argued that this language accurately conveys an aspect of
Dumas' work, but not even his worst detractors would pretend that there
is nothing more to it than that. Still less acceptable, however, than
the language of this Victorian translation is the huge number of
omissions and bowdlerizations of Dumas' text. The latter include part of
Franz's opium dream at the end of Chapter {\small XXXI},
some of the dialogue between Villefort and Madame Danglars in
Chapter {\small LXVII}, and several parts of
Chapter {\small XCVII}, on Eug\'enie and Louise's flight to
Belgium. In some cases the changes are so slight as to be quite hard to
detect. In the description of Eug\'enie at the opera
(Chapter {\small LIII}) for example, Dumas remarks that,
if one could reproach her with anything, it was that, both in her
upbringing and her appearance, `she seemed rather to belong to another
sex'. The English translator renders this: `As for her attainments, the
only fault to be found with them was \ldots{} that they were somewhat
too erudite and masculine for so young a person' (p. 542)! At the end of
Chapter {\small XCVII}, the translation (p. 950) simply
omits the few lines of dialogue where Dumas has Eug\'enie say that
`\emph{le rapt est bel et bien consomm\^e}' -- where the word \emph{rapt}
(`abduction') has a rather too overtly sexual connotation. Similarly,
earlier in the same chapter, where Eug\'enie jokes that anyone would think
she was `abducting' (\emph{enl\`eve}) Louise -- another word used almost
exclusively of a man with a woman -- the translator prefers the more
neutral phrase `carrying me off' and omits altogether Louise's remark
that Eug\'enie is `a real Amazon'. Another anonymous translation (Dent,
1894) refers to `the escape' rather than `the abduction' -- which makes
nonsense of Louise's reply that it is not a true abduction since it has
been accomplished without violence.

What may be more surprising than these concessions to the prudery of the
age is that the Victorian translators left in as much as they did. And
the omissions are by no means all to do with sexual matters. At the
start of Chapter {\small XXXIV}, for example, the
translator decides to spare us the description of the route taken
through Rome by Albert and Franz on their way to the Colosseum (though
the 1894 translator restores it). A whole paragraph analysing the
character of M. de Villefort at the start of
Chapter {\small XLVIII} is cut out; almost a whole page of
dialogue between Albert and Monte Cristo, on horses, in
Chapter {\small LXXXV} is cavalierly omitted (part was
restored by the translator of 1894); and so on. This is only a tiny
sample of what is, in reality, a vast number of phrases omitted, and
occasionally mistranslated.

What we see here, interestingly enough, is a stage in the process of
transforming Dumas' text into something simpler, less complex, less rich
in allusions, but more concentrated in plot and action. The 1846
translator already has an idea of what kind of novel this is, and that
dictates what he, or she, can afford to omit: travelogue, classical
references, sexual and psychological analysis, and so on. None of these
is essential to the plot of a thriller, and if some of them will
embarrass English readers, then why leave them in? The only problem is
that, nearly 150 years later, we do not have quite the same idea of what
is and what is not important. It was high time to go back to Dumas,
entire and unexpurgated.

As the basis for my translation, I have used the edition by Schopp,
quoted above, and the three-volume edition in the \emph{Livre de Poche}
(1973). Both of these use an arrangement of chapters which differs
slightly from that in the nineteenth-century English translations. I
have followed the \emph{Livre de Poche} in not changing Dumas' `errors'
of chronology etc. in the text as Schopp does; instead I have pointed
out the more important ones in the notes. I owe a debt to Schopp and to
Coward's edition in the World's Classics series for some of the
information in the notes.

On the broader question of translation, I have tried above all to
produce a version that is accurate and readable. A great deal of
nonsense is written about translation, particularly by academics who
approach it either as a terrain for theoretical debate or, worse still,
as a moral issue: `the translator must always be faithful to his
original,' Leonard Tancock wrote, oddly assuming that translation is a
masculine activity, even though on this occasion he was prefacing Nancy
Mitford's translation of \emph{La Princesse de Cl\`eves} (Penguin, 1978).
` \ldots{} he has no right whatever to take liberties with it \ldots{}
Nor has he any right to try to smooth the reader's path by the omission
of ``dull'' bits, short-circuitings, explanatory additions, radical
transferences or changes of order.' Why? And who says? Is it the reader
who is demanding this perfection, this absence of explanatory additions,
and so on?

Such academic theorists insist that a translation must read like a
translation -- it is somehow immoral to conceal the process that has
gone into making it. `Ordinary' readers usually demand the opposite, and
reviewers in quite respectable papers sometimes show little appreciation
of what the process means and involves: `Not all of this material works
in translation,' said one serious review of a book by Umberto Eco; and
another: ` \ldots{} the stories {[}of Viktoria Tokareva{]} are well
served by their translator, who hardly ever gets in the way'.

In philosophical terms I am quite willing to admit the impossibility of
translation, while still having in practical terms to engage in it and
to believe that everything must, to some extent, be translatable. I feel
no obligation to avoid smoothing the reader's path and none, on the
other hand, to `getting in the way' from time to time. Above all, I want
to convey some of the pleasure of reading Dumas to those who cannot do
so in the original language and, through my one, particular version
(since no translation can ever be definitive), to reveal aspects of his
work that are not to be found in any of the other existing versions.
This is a new translation and consequently a new interpretation of a
great -- and great popular -- novel. If nothing else, most people would
surely agree that it is long overdue.

%------------------------------------------------
% Main matter
%------------------------------------------------
\mainmatter

\addcontentsline{toc}{book}{The Count of Monte Cristo}
\book*{The Count of \\ Monte Cristo}

\addcontentsline{toc}{part}{Part 1}
\part*{Part 1}

\mychapter{Marseille -- Arrival}

On February 24, 1815, the lookout at Notre-Dame de la Garde signalled
the arrival of the three-master \emph{Pharaon}, coming from Smyrna,
Trieste and Naples. As usual, a coastal pilot immediately left the port,
sailed hard by the Ch\^ateau d'If, and boarded the ship between the Cap de
Morgiou and the island of Riou.

At once (as was also customary) the terrace of Fort
Saint-Jean\textsuperscript{1} was thronged with
onlookers, because the arrival of a ship is always a great event in
Marseille, particularly when the vessel, like the \emph{Pharaon}, has
been built, fitted out and laded in the shipyards of the old port and
belongs to an owner from the town.

Meanwhile the ship was drawing near, and had successfully negotiated the
narrows created by some volcanic upheaval between the islands of
Calseraigne and Jarre; it had rounded Pom\`egue and was proceeding under
its three topsails, its outer jib and its spanker, but so slowly and
with such melancholy progress that the bystanders, instinctively sensing
some misfortune, wondered what accident could have occurred on board.
Nevertheless, those who were experts in nautical matters acknowledged
that, if there had been such an accident, it could not have affected the
vessel itself, for its progress gave every indication of a ship under
perfect control: the anchor was ready to drop and the bowsprit shrouds
loosed. Next to the pilot, who was preparing to guide the \emph{Pharaon}
through the narrow entrance to the port of Marseille, stood a young man,
alert and sharp-eyed, supervising every movement of the ship and
repeating each of the pilot's commands.

One of the spectators on the terrace of Fort Saint-Jean had been
particularly affected by the vague sense of unease that hovered among
them, so much so that he could not wait for the vessel to come to land;
he leapt into a small boat and ordered it to be rowed out to the
\emph{Pharaon}, coming alongside opposite the cove of La R\'eserve. When
he saw the man approaching, the young sailor left his place beside the
pilot and, hat in hand, came and leant on the bulwarks of the ship.

He was a young man of between eighteen and twenty, tall, slim, with fine
dark eyes and ebony-black hair. His whole demeanour possessed the calm
and resolve peculiar to men who have been accustomed from childhood to
wrestle with danger.

`Ah, it's you, Dant\`es!' the man in the boat cried. `What has happened,
and why is there this air of dejection about all on board?'

`A great misfortune, Monsieur Morrel!' the young man replied. `A great
misfortune, especially for me: while off Civita Vecchia, we lost our
good Captain Lecl\`ere.'

`And the cargo?' the ship owner asked brusquely.

`It has come safe to port, Monsieur Morrel, and I think you will be
content on that score. But poor Captain Lecl\`ere \ldots'

`What happened to him, then?' the shipowner asked, visibly relieved. `So
what happened to the good captain?'

`He is dead.'

`Lost overboard?'

`No, Monsieur, he died of an apoplectic fever, in terrible agony.' Then,
turning back to his crew, he said: `Look lively, there! Every man to his
station to drop anchor!'

The crew obeyed. As one man, the eight or ten sailors of which it was
composed leapt, some to the sheets, others to the braces, others to the
halyards, others to the jib, and still others to the brails. The young
sailor glanced casually at the start of this operation and, seeing that
his orders were being carried out, prepared to resume the conversation.

`But how did this misfortune occur?' the shipowner continued, picking it
up where the young man had left off.

`By heaven, Monsieur, in the most unexpected way imaginable: after a
long conversation with the commander of the port, Captain Lecl\`ere left
Naples in a state of great agitation. Twenty-four hours later, he was
seized with fever and, three days after that, he was dead \ldots{} We
gave him the customary funeral and he now rests, decently wrapped in a
hammock, with a thirty-six-pound cannonball at his feet and another at
his head, off the island of Giglio. We've brought his medal and his
sword back for his widow. Much good it did him,' the young man
continued, with a melancholy smile, `to fight the war against the
English for ten years -- only to die at last, like anyone else, in his
bed.'

`Dammit, Monsieur Edmond, what do you expect?' said the shipowner, who
appeared to be finding more and more to console him in his grief. `We
are all mortal. The old must give way to the young, or else there would
be no progress or promotion. As long as you can assure me that the cargo
\ldots'

`All is well with it, Monsieur Morrel, I guarantee you. If you take my
advice, you will not discount this trip for a profit of 25,000 francs.'

Then, as they had just sailed past the Round Tower, the young sailor
cried: `Furl the topmast sails, the jib and the spanker! Look lively!'

The order was obeyed with almost as much dispatch as on a man-o'-war.

`Let go and brail all!'

At this last command, all the sails were lowered and the progress of the
ship became almost imperceptible, driven only by the impetus of its
forward motion.

`And now, if you would like to come aboard, Monsieur Morrel,' Dant\`es
said, observing the owner's impatience, `I see your
supercargo,\textsuperscript{2} Monsieur Danglars,
coming out of his cabin. He will give you all the information that you
desire. As for me, I must see to the mooring and put the ship in
mourning.'

The owner did not need asking twice. He grasped hold of a line that
Dant\`es threw to him and, with an agility that would have done credit to
a seaman, climbed the rungs nailed to the bulging side of the ship,
while Dant\`es went back to his post and left the conversation to the man
he had introduced as Danglars: the latter was indeed emerging from his
cabin and coming across to the shipowner.

This new arrival was a man, twenty-five to twenty-six years old,
somewhat sombre in appearance, obsequious towards his superiors and
insolent to his subordinates; hence, even apart from the label of
supercargo, which always in itself causes aversion among sailors, he was
generally as much disliked by the crew as Dant\`es was loved by them.

`Well, Monsieur Morrel,' said Danglars, `you have heard the bad news, I
suppose?'

`Yes, yes, poor Captain Lecl\`ere! He was a fine and upright man!'

`And above all an excellent sailor, weathered between the sea and the
heavens, as was proper in a man responsible for looking after the
interests of so important a firm as Morrel and Son,' Danglars replied.

`Even so,' the shipowner replied, watching Dant\`es while he searched for
his mooring. `Even so, I think one need not be a seaman of such long
experience as you say, Danglars, to know the business: there is our
friend Edmond going about his, it seems to me, like a man who has no
need to ask advice of anybody.'

`Indeed,' said Danglars, casting a sidelong glance at Dant\`es with a
flash of hatred in his eyes. `Yes, indeed, he is young and full of
self-confidence. The captain was hardly dead before he had taken command
without asking anyone, and made us lose a day and a half on the island
of Elba, instead of returning directly to Marseille.'

`As far as taking command of the ship is concerned,' said the owner,
`that was his duty as first mate. As for losing a day and a half at
Elba, he was in the wrong, unless there was some damage to the ship that
needed repairing.'

`The ship was in as good shape as I am, and as good as I hope you are,
Monsieur Morrel. That day and a half was lost on a whim, for nothing
other than the pleasure of going ashore.'

`Dant\`es,' the owner said, turning towards the young man. `Would you come
here.'

`Your pardon, Monsieur,' Dant\`es said. `I shall be with you in an
instant.' Then, to the crew, he called: `Drop anchor!'

The anchor was immediately lowered and the chain ran out noisily. Dant\`es
stayed at his post, even though the pilot was there, until the last
operation had been carried out, then ordered: `Lower the pennant and the
flag to half-mast, unbrace the yards!'

`You see,' Danglars said. `I do believe he thinks himself captain
already.'

`So he is, in effect,' said the owner.

`Yes, apart from your signature and that of your partner, Monsieur
Morrel.'

`By gad, why shouldn't we leave him in the job?' said the owner. `He is
young, I grant you, but he seems made for it and very experienced in his
work.'

A cloud passed across Danglars' brow.

`Excuse me, Monsieur Morrel,' Dant\`es said as he came over. `Now that the
ship is moored, I am entirely at your disposal: I think you called me?'

Danglars took a step back.

`I wanted to ask why you stopped on the island of Elba.'

`I don't know, Monsieur. It was to carry out a last order from Captain
Lecl\`ere, who gave me, on his deathbed, a packet for Marshal
Bertrand.'\textsuperscript{3}

`Did you see him, Edmond?'

`Whom?'

`The Grand Marshal.'

`Yes.'

Morrel looked about him and drew Dant\`es aside.

`And how is the emperor?' he asked, earnestly.

`He is well, as far as I can judge by my own eyes.'

`So you saw the emperor, too, did you?'

`He came to visit the marshal while I was there.'

`And did you speak to him?'

`It was he, Monsieur, who spoke to me,' Dant\`es said, smiling.

`And what did he say?'

`He asked me about the ship, the time of its departure for Marseille,
the route it had taken and the cargo we were carrying. I think that, had
it been empty and I the master of it, he intended to buy it; but I told
him that I was only the first mate and that the ship belonged to the
firm of Morrel and Son. ``Ah, yes!'' he said. ``I know them. The Morrels
have been shipowners from father to son, and there was Morrel who served
in the same regiment as I did, when I was garrisoned at Valence.'' '

`By heaven, that's a fact!' the shipowner cried, with delight. `It was
Policar Morrel, my uncle, who later made captain. Dant\`es, tell my uncle
that the emperor remembered him, and you will bring tears to the old
trooper's eyes. Come, come, now,' he went on, putting a friendly arm
across the young man's shoulders, `you did well to follow Captain
Lecl\`ere's instructions and stop on Elba; even though, if it were known
that you gave a packet to the marshal and spoke to the emperor, you
might be compromised.'

`How could it compromise me, Monsieur?' said Dant\`es. `I don't even know
what I was carrying, and the emperor only asked me the same questions
that he would have put to anyone else. But please excuse me,' he
continued. `The health authorities and the Customs are coming on board.
With your permission?'

`Of course, of course, my dear Dant\`es, carry on.'

The young man went off and, as he did so, Danglars returned.

`So,' he asked, `it appears that he gave you good reason for stopping
off at Porto Ferrajo?'

`Excellent reason, my dear Danglars.'

`I am pleased to hear it,' the other replied. `It is always distressing
to see a comrade fail in his duty.'

`Dant\`es did his duty,' the shipowner answered, `and there is no more to
be said. It was Captain Lecl\`ere who ordered him to put into port.'

`Speaking of Captain Lecl\`ere, did he not give you a letter from him?'

`Who?'

`Dant\`es.'

`Not to me! Was there one?'

`I believe that, apart from the packet, Captain Lecl\`ere entrusted him
with a letter.'

`Which packet are you referring to, Danglars?'

`The same that Dant\`es delivered when we stopped at Porto Ferrajo.'

`And how did you know that he had a packet to deliver at Porto Ferrajo?'

Danglars blushed. `I was passing by the door of the captain's cabin,
which was partly open, and I saw him handing the packet and a letter to
Dant\`es.'

`He did not mention it,' said the owner. `But if he has such a letter,
he will give it to me.'

Danglars thought for a moment.

`In that case, Monsieur Morrel,' he said, `I beg you to say nothing
about it to Dant\`es. I must have been mistaken.'

At that moment the young man came back and Danglars left them.

`Now, my dear Dant\`es, are you free?' the owner asked.

`Yes, Monsieur.'

`It did not take long.'

`No, I gave the Customs a list of our cargo; as for the port
authorities, they sent a man with the coastal pilot, and I handed our
papers over to him.'

`So you have nothing more to do here?'

Dant\`es cast a rapid glance about him. `No, everything is in order,' he
said.

`Then you can come and take dinner with us?'

`Please, Monsieur Morrel, I beg you to excuse me, but the first thing I
must do is to visit my father. Nonetheless, I am most grateful for the
honour you do me.'

`That's proper, Dant\`es, very proper. I know that you are a good son.'

`And \ldots' Dant\`es asked, somewhat hesitantly, `as far as you know,
he's in good health, my father?'

`I do believe so, my dear Edmond, though I have not seen him.'

`Yes, he stays shut up in his little room.'

`Which at least proves that he lacked nothing while you were away.'

Dant\`es smiled.

`My father is a proud man, Monsieur, and even if he were short of
everything, I doubt if he would have asked for help from anyone in the
world, except God.'

`Now, when you have done that, we can count on your company.'

`I must beg you once more to excuse me, Monsieur Morrel, but after that
first visit, there is another that is no less important to me.'

`Ah, Dant\`es, that's true; I was forgetting that there is someone in Les
Catalans who must be expecting you with no less impatience than your
father -- the lovely Merc\'ed\`es.'

Dant\`es smiled.

`Ah, ha,' said the owner, `now I understand why she came three times to
ask me for news of the \emph{Pharaon}. Dash it, Edmond! You're a lucky
fellow, to have such a pretty mistress.'

`She is not my mistress, Monsieur,' the young sailor said gravely. `She
is my fianc\'ee.'

`It sometimes amounts to the same thing,' the owner said, with a
chuckle.

`Not for us, Monsieur,' Dant\`es replied.

`Come, come, my dear Edmond,' the other continued. `Don't let me detain
you. You have looked after my business well enough for me to give you
every opportunity to look after your own. Do you need any money?'

`No, Monsieur, I have all my salary from the trip -- that is, nearly
three months' pay.'

`You manage your affairs well, my boy.'

`You might add that my father is a poor man, Monsieur Morrel.'

`Yes, indeed, I know you are a good son to him. So: go and see your
father. I, too, have a son and I should bear a grudge against the man
who kept him away from me, after a three-month voyage.'

`May I take my leave, then?' the young man said, with a bow.

`Yes, if you have nothing more to say to me.'

`No.'

`When Captain Lecl\`ere was dying, he did not give you a letter for me?'

`It would have been impossible for him to write one, Monsieur. But that
reminds me: I wanted to ask you for a fortnight's leave.'

`To get married?'

`Firstly, then to go to Paris.'

`Very well! Have as much time as you want, Dant\`es. It will take us a
good six weeks to unload the vessel and we shall hardly be ready to put
to sea again within three months \ldots{} In three months' time,
however, you must be there. The \emph{Pharaon},' the shipowner
continued, putting a hand across the young sailor's shoulders, `cannot
set sail without its captain.'

`Without its captain!' Dant\`es cried, his eyes lighting up with joy. `Be
very careful what you are saying, Monsieur, because you have just
touched on the most secret of my heart's desires. Can it be that you
intend to appoint me captain of the \emph{Pharaon}?'

`If it was up to me alone, I should grasp your hand, my dear Dant\`es, and
say to you: ``the matter is settled!'' But I have a partner, and you
know the Italian proverb: \emph{chi ha compagno, ha
padrone}.\textsuperscript{4} But, at least, we
are half-way there, since you already have one of the two votes you
need. Leave it to me to get you the other, and I shall do my best.'

`Oh, Monsieur Morrel!' the young sailor cried, with tears in his eyes,
grasping the shipowner's hands. `Monsieur Morrel, I thank you, on behalf
of my father and of Merc\'ed\`es.'

`Fine, Edmond, fine! There is a God in heaven who looks after honest
folk. Go and see your father, go and see Merc\'ed\`es, then when that's
done, come and see me.'

`But don't you want me to accompany you back to land?'

`No, thank you. I shall stay here to settle my accounts with Danglars.
Were you happy with him during the voyage?'

`It depends on what you understand by that question, Monsieur. If you
mean, as a good companion, no, because I think that he has not liked me
since the day when I had the folly, after a trifling dispute between us,
to suggest that we should stop for ten minutes on the isle of Monte
Cristo to settle the matter. It was wrong of me to propose that, and he
was right to refuse. If you are asking me about him as a supercargo, I
think there is nothing to say, and that you will be satisfied with the
manner in which his duties have been carried out.'

`Come now, Dant\`es,' the shipowner asked, `if you were captain of the
\emph{Pharaon}, would you be pleased to keep Danglars?'

`Whether as captain or as first mate, Monsieur Morrel,' Dant\`es replied,
`I shall always have the highest regard for those who enjoy the
confidence of my owners.'

`Well, well, Dant\`es, you are clearly a fine lad, in every respect. Let
me detain you no longer, for I can see that you are on tenterhooks.'

`I may take my leave?' asked Dant\`es.

`Go on, I'm telling you.'

`Will you permit me to use your boat?'

`Take it.'

`Au revoir, Monsieur Morrel, and thank you a thousand times.'

`Au revoir, dear Edmond, and good luck!'

The young sailor leapt into the boat, seated himself in the stern and
gave the order to row across to the Canebi\`ere. Two sailors immediately
bent over their oars and the vessel proceeded as fast as it could, among
the thousand small boats that obstruct the sort of narrow alleyway
leading, between two lines of ships, from the harbour entrance to the
Quai d'Orl\'eans.

The shipowner looked after him, smiling, until the boat touched land and
he saw him leap on to the cobbled quay, where he was instantly lost in
the variegated crowd that, from five in the morning until nine in the
evening, throngs the famous street known as La Canebi\`ere: the modern
inhabitants of this old Phocean colony are so proud of it that they
proclaim, with all the seriousness in the world, in that accent which
gives such savour to everything they say: `If Paris had the Canebi\`ere,
Paris would be a little Marseille.'

Turning, the shipowner saw Danglars standing behind him, apparently
awaiting orders but in reality, like him, watching the young sailor's
departure. Yet there were very different expressions in these two pairs
of eyes following the one man.

\bigskip

\bigskip

\mychapter{Father and Son}

We shall leave Danglars, gripped by the demon of hatred, trying to
poison the shipowner's ear with some malicious libel against his
comrade, and follow Dant\`es who, after walking along the Canebi\`ere, took
the Rue de Noailles, entered a small house on the left side of the
All\'ees de Meilhan and hastened up the four flights of a dark stairway.
There, holding the banister with one hand, while the other repressed the
beating of his heart, he stopped before a half-open door through which
he could see to the back of a small room.

In this room lived Dant\`es' father.

News of the arrival of the \emph{Pharaon} had not yet reached the old
man who was standing on a chair, engaged with trembling hands in pinning
up some nasturtiums and clematis that climbed across the trellis outside
his window. Suddenly, he felt himself grasped around the waist and a
well-known voice exclaim behind him: `Father! My dear father!'

The old man cried out and turned around; then, seeing his son, fell into
his arms, pale and trembling.

`What is it, father?' the young man exclaimed, with concern. `Are you
unwell?'

`No, no, dear Edmond, my son, my child. No, but I was not expecting you
-- and the joy, the shock of seeing you like this, unexpectedly \ldots{}
Oh, heavens! It is too much for me!'

`Now, then, father, calm yourself! I am really here! They always say
that joy cannot harm you, which is why I came in without warning. Come
now, smile; don't look at me like that, with those wild eyes. I am back
and there is happiness in store for us.'

`I'm pleased to hear it, my boy,' the old man continued. `But what
happiness? Are you going to stay with me from now on? Come, tell me
about your good fortune!'

`God forgive me,' the young man said, `for rejoicing at good fortune
which has brought grief to the family of another. But, God knows, I
never wished for it; it has happened, and I do not have the heart to
grieve at it. Our good Captain Lecl\`ere is dead, father, and it seems
likely that, thanks to Monsieur Morrel's support, I shall have his
command. Do you understand, father? A captain at twenty! With a salary
of a hundred \emph{louis}\textsuperscript{1} and
a share in the profits! Isn't that better than a poor sailor like myself
could expect?'

`Yes, my son, yes,' said the old man. `This is indeed a stroke of luck.'

`So I want you to have a little house, with the first money I earn, and
a garden to grow your clematis, your nasturtiums and your honeysuckle
\ldots{} But what's wrong, father? You look ill!'

`An instant, don't worry! It is nothing.' And, his strength failing him,
he leant back.

`Father!' cried the young man. `Come, have a glass of wine; it will
revive you. Where do you keep your wine?'

`No, thank you, don't bother to look for it; there is no need,' he
replied, trying to restrain his son.

`Yes, indeed there is, father. Show me it.' He opened one or two
cupboards.

`It's a waste of time \ldots' the old man said. `There is no wine left.'

`What! No wine!' Dant\`es said, paling in turn as he looked from the old
man's sunken and livid cheeks to the empty cupboards. `What! You have no
wine left? Have you been short of money, father?'

`I am short of nothing, now that you are here,' said the old man.

`But I left you two hundred francs,' Dant\`es stammered, wiping the sweat
from his brow, `two months ago, as I was leaving.'

`Yes, yes, Edmond, so you did; but when you left you forgot a small debt
to my neighbour Caderousse. He reminded me of it and said that if I did
not settle it on your behalf, he would go and reclaim it from Monsieur
Morrel. So, you understand, I was afraid that it might do you some
harm.'

`And?'

`And I paid it.'

`But,' Dant\`es exclaimed, `I owed Caderousse a hundred and forty francs!'

`Yes,' the old man mumbled.

`And you paid them out of the two hundred francs that I left you?'

His father nodded.

`Which means that you lived for three months on sixty francs!' the young
man exclaimed.

`You know how small my needs are.'

`Oh, heaven, heaven, forgive me!' Edmond cried, falling on his knees in
front of the old man.

`What are you doing?'

`Ah! You have broken my heart!'

`Pah! You are here,' the old man said, with a smile. `All is forgotten,
because all is well.'

`Yes, here I am,' said the young man. `Here I am with a fine future and
a little money. Here, father,' he said, `take it, take it and send out
for something immediately.'

He emptied the contents of his pockets on the table: a dozen gold coins,
five or six five-franc pieces and some small change.

Old Dant\`es' face lit up.

`Whose is that?' he asked.

`Mine! Thine! Ours, of course! Take it, buy some food and enjoy
yourself. There will be more tomorrow.'

`Gently, gently,' the old man said, smiling. `If you don't mind, I shall
go easy on your money: if people see me buying too many things at once,
they will think that I had to wait for you to come back before I went
shopping.'

`Do as you think best, but first of all, father, get yourself a
housemaid: I don't want you to live on your own from now on. I have some
contraband coffee and some excellent tobacco in a little chest in the
hold. You will have it tomorrow. But, hush! Someone is coming.'

`That will be Caderousse, who has learned of your arrival and is no
doubt coming to welcome you back.'

`There's a fellow who says one thing and thinks another,' Edmond
muttered. `No matter. He is a neighbour who has helped us in the past,
so let him come in.'

Just as Edmond finished saying this under his breath, the black, bearded
head of Caderousse appeared on the landing, framed in the outer door. A
man of twenty-five or twenty-six years of age, he was holding a piece of
cloth which, being a tailor, he was about to fashion into the lining of
a jacket.

`You're back again, then, Edmond?' he said, with a thick Marseille
accent and a broad smile, revealing teeth as white as ivory.

`As you can see, neighbour, and entirely at your service,' Dant\`es
replied, this polite formula barely disguising his coldness towards the
man.

`Thank you, thank you. Fortunately, I need nothing; in fact, it is
sometimes others who need me.' Dant\`es bridled. `I am not saying that for
you, my boy. I lent you money and you returned it. That's how things are
done between good neighbours, and we're quits.'

`We are never quits towards those who have done us a favour,' said
Dant\`es. `Even when one ceases to owe them money, one owes them
gratitude.'

`There is no sense in speaking of that: what's past is past. Let's talk
about your happy return, young man. I just happened to go down to the
harbour to fetch some brown cloth, when I met our friend Danglars.
``You're in Marseille?'' I exclaimed. ``Yes, as you see.'' ``I thought
you were in Smyrna.'' ``It could well be, because I have just come back
from there.'' ``And where is young Edmond, then?'' ``At his father's, I
suppose,'' Danglars told me. So I came at once,' Caderousse concluded,
`to have the pleasure of shaking the hand of a friend.'

`Dear Caderousse,' the old man said. `He is so fond of us.'

`Indeed, I am, and I hold you in all the greater esteem, since honest
people are so rare! But it seems you have come into money, my boy?' the
tailor went on, glancing at the handful of gold and silver that Dant\`es
had emptied on to the table.

The young man observed a flash of greed light up his neighbour's dark
eyes. `Heavens, no!' he said casually. `That money is not mine. I was
just telling my father that I was afraid he might have wanted for
something while I was away and, to reassure me, he emptied his purse on
the table. Come, father,' he continued. `Put that money back in your
pocket -- unless, of course, our neighbour needs some for himself, in
which case it is at his disposal.'

`Indeed not, my boy,' said Caderousse. `I need nothing and, thank God,
my business holds body and soul together. Keep your money, keep it; one
can never have too much. Still, I am obliged for your offer, as much as
if I had taken advantage of it.'

`It was well meant,' said Dant\`es.

`I don't doubt that it was. So, I learn that you are on good terms with
Monsieur Morrel, sly one that you are?'

`Monsieur Morrel has always been very good to me,' Dant\`es answered.

`In that case, you were wrong to refuse dinner with him.'

`What do you mean: refuse dinner?' Old Dant\`es asked. `Did he invite you
to dinner?'

`Yes, father,' said Edmond, smiling at his father's astonishment on
learning of this high honour.

`So why did you refuse, son?' the old man asked.

`So that I could come straight back here, father,' the young man
answered. `I was anxious to see you.'

`He must have been put out by it, that good Monsieur Morrel,' Caderousse
remarked. `When one hopes to be made captain, it is a mistake to get on
the wrong side of one's owner.'

`I explained the reason for my refusal and I hope he understood it.'

`Even so, to be promoted to captain, one must flatter one's bosses a
little.'

`I expect to become captain without that,' Dant\`es retorted.

`So much the better! All your old friends will be pleased for you and I
know someone over there, behind the Citadelle de Saint-Nicholas, who
will not be unhappy about it, either.'

`Merc\'ed\`es?' the old man said.

`Yes, father,' Dant\`es resumed. `And, with your permission, now that I've
seen you, now that I know you are well and that you have all you need, I
would like to ask your leave to go and visit Les Catalans.'

`Go, child,' Old Dant\`es said. `And may God bless you as much in your
wife as He has blessed me in my son.'

`His wife!' said Caderousse. `Hold on, old man, hold on! As far as I
know, she's not that yet!'

`No,' Edmond replied, `but in all probability she soon will be.'

`Never mind,' said Caderousse, `never mind. You have done well to hurry
back, my boy.'

`Why?'

`Because Merc\'ed\`es is a beautiful girl, and beautiful girls are never
short of admirers, especially that one: there are dozens of them after
her.'

`Really?' Edmond said with a smile, not entirely concealing a hint of
unease.

`Oh, yes,' Caderousse continued, `and some with good prospects, too.
But, of course, you are going to be a captain, so she'll be sure not to
refuse you.'

`By which you mean,' Dant\`es said, smiling, but barely concealing his
anxiety, `that if I were not a captain \ldots'

`Ah! Ah!' said Caderousse.

`Come, now,' the young man said. `I have a better opinion than you of
women in general, and Merc\'ed\`es in particular, and I am persuaded that,
whether I were a captain or not, she would remain faithful to me.'

`So much the better! When one is going to get married, it is always a
good thing to have faith. But enough of that. Take my advice, lad: don't
waste any time in telling her of your return and letting her know about
your aspirations.'

`I am going at once,' said Edmond.

He embraced his father, nodded to Caderousse and left.

Caderousse stayed a moment longer, then, taking his leave of the elder
Dant\`es, followed the young man down and went to find Danglars who was
waiting for him on the corner of the Rue Senac.

`Well?' Danglars asked. `Did you see him?'

`I have just left them,' said Caderousse.

`And did he talk about his hope of being made captain?'

`He spoke of it as though he had already been appointed.'

`Patience!' Danglars said. `It seems to me that he is in rather too much
of a hurry.'

`Why, it seems Monsieur Morrel has given him his word.'

`So he is pleased?'

`He is even insolent about it. He has already offered me his services,
like some superior personage; he wanted to lend me money, like some
banker or other.'

`You refused?'

`Indeed I did, though I could well have accepted, since I am the one who
gave him the first silver coins he ever had in his hands. But now
Monsieur Dant\`es has no need of anyone: he is going to be a captain.'

`Huh!' said Danglars. `He's not one yet.'

`My God, it would be a fine thing indeed if he wasn't,' said Caderousse.
`Otherwise there will be no talking to him.'

`If we really want,' said Danglars, `he will stay as he is, and perhaps
even become less than he is.'

`What do you mean?'

`Nothing, I was talking to myself. Is he still in love with the
beautiful Catalan?'

`Madly. He has gone there now; but, unless I am gravely mistaken, he
will not find things altogether to his liking.'

`Explain.'

`What does it matter?'

`This is more important than you may think. You don't like Dant\`es, do
you?'

`I don't like arrogance.'

`Well, then: tell me what you know about the Catalan woman.'

`I have no positive proof, but I have seen things, as I said, that make
me think the future captain will not be pleased with what he finds
around the Chemin des Vieilles-Infirmeries.'

`What have you seen? Come on, tell me.'

`Well, I have observed that every time Merc\'ed\`es comes into town, she is
accompanied by a large Catalan lad, with black eyes, ruddy cheeks, very
dark in colour and very passionate, whom she calls ``my cousin''.'

`Ah, indeed! And do you think this cousin is courting her?'

`I imagine so: what else does a fine lad of twenty-one do to a pretty
girl of seventeen?'

`And you say that Dant\`es has gone to Les Catalans?'

`He left before me.'

`Suppose we were to go in the same direction, stop in the R\'eserve and,
over a glass of La Malgue wine, learn what we can learn.'

`Who would tell us anything?'

`We shall be on the spot and we'll see what has happened from Dant\`es'
face.'

`Let's go then,' said Caderousse. `But you are paying?'

`Certainly,' Danglars replied.

The two of them set off at a brisk pace for the spot they had mentioned
and, when they arrived, called for a bottle and two glasses.

Old Pamphile had seen Dant\`es go by less than two minutes before. Certain
that he was in Les Catalans, they sat under the budding leaves of the
plane-trees and sycamores, in the branches of which a happy band of
birds was serenading one of the first fine days of spring.

\bigskip

\bigskip

\mychapter{Les Catalans}

A hundred yards away from the place where the two friends, staring into
the distance with their ears pricked, were enjoying the sparkling wine
of La Malgue, lay the village of Les Catalans, behind a bare hillock
ravaged by the sun and the mistral.

One day, a mysterious group of colonists set out from Spain and landed
on this spit of land, where it still resides today. No one knew where
they had come from or what language they spoke. One of the leaders, who
understood Proven\c{c}al, asked the commune of Marseille to give them this
bare and arid promontory on to which, like the sailors of Antiquity,
they had drawn up their boats. The request was granted and, three months
later, a little village grew up around the twelve or fifteen boats that
brought these gypsies of the sea.

The same village, built in a bizarre and picturesque manner that is
partly Moorish and partly Spanish, is the one that can be seen today,
inhabited by the descendants of those men, who speak the language of
their forefathers. For three or four centuries they have remained
faithful to the little promontory on which they first landed, clinging
to it like a flock of seabirds, in no way mixing with the inhabitants of
Marseille, marrying among themselves and retaining the habits and dress
of their motherland, just as they have retained its tongue.

The reader must follow us along the only street of the little village
and enter one of those houses, to the outside of which the sunlight has
given that lovely colour of dead leaves which is peculiar to the
buildings of the country; with, inside, a coat of whitewash, the only
decoration of a Spanish \emph{posada}.

A lovely young girl with jet-black hair and the velvet eyes of a
gazelle, was standing, leaning against an inner wall, rubbing an
innocent sprig of heather between slender fingers like those on a
classical statue, and pulling off the flowers, the remains of which were
already strewn across the floor. At the same time, her arms, naked to
the elbow, arms that were tanned but otherwise seemed modelled on those
of the Venus of Arles, trembled with a sort of feverish impatience, and
she was tapping the ground with her supple, well-made foot, revealing a
leg that was shapely, bold and proud, but imprisoned in a red cotton
stocking patterned in grey and blue lozenges.

A short distance away, a tall young man of between twenty and twenty-two
was sitting on a chair, rocking it fitfully on two legs while supporting
himself on his elbow against an old worm-eaten dresser and watching her
with a look that combined anxiety with irritation. His eyes were
questioning, but those of the young woman, firm and unwavering,
dominated their conversation.

`Please, Merc\'ed\`es,' the man said. `Easter is coming round again; it's
the time for weddings. Give me your answer!'

`You have had it a hundred times, Fernand, and you really must like
torturing yourself, to ask me again.'

`Well, repeat it, I beg you, repeat it once more so that I can come to
believe it. Tell me, for the hundredth time, that you reject my love,
even though your mother approves of me. Convince me that you are
prepared to trifle with my happiness and that my life and my death are
nothing to you. My God, my God! To dream for ten years of being your
husband, Merc\'ed\`es, and then to lose that hope which was the sole aim of
my existence!'

`I, at least, never encouraged you in that hope, Fernand,' Merc\'ed\`es
replied. `You cannot accuse me of having, even once, flirted with you.
I've said repeatedly: ``I love you like a brother, but never demand
anything more from me than this fraternal love, because my heart belongs
to another.'' Isn't that what I have always told you, Fernand?'

`Yes, Merc\'ed\`es, I know,' the young man replied. `Yes, you have always
been laudably, and cruelly, honest with me. But are you forgetting that
it is a sacred law among the Catalans only to marry among themselves?'

`You are wrong, Fernand, it is not a law, but a custom, nothing more;
and I advise you not to appeal to that custom on your behalf. You have
been chosen for conscription, Fernand, and the freedom that you now
enjoy is merely a temporary reprieve: at any moment you might be called
up to serve in the army. Once you are a soldier, what will you do with
me -- I mean, with a poor orphan girl, sad and penniless, whose only
possession is a hut, almost in ruins, in which hang a few worn nets, the
paltry legacy that was left by my father to my mother, and by my mother
to me? Consider, Fernand, that in the year since she died, I have
virtually lived on charity! Sometimes you pretend that I am of some use
to you, so that you can be justified in sharing your catch with me. And
I accept, Fernand, because you are the son of one of my father's
brothers, because we grew up together and, beyond that, most of all,
because it would hurt you too much if I were to refuse. But I know full
well that the fish I take to the market, which bring me the money to buy
the hemp that I spin -- I know, full well, Fernand, that it is charity.'

`What does it matter, Merc\'ed\`es, poor and alone as you are, when you suit
me thus better than the daughter of the proudest shipowner or the
richest banker in Marseille? What do people like us need? An honest wife
and a good housekeeper. Where could I find better than you on either
score?'

`Fernand,' Merc\'ed\`es replied, shaking her head, `one is not a good
housekeeper and one cannot promise to remain an honest woman when one
loves a man other than one's husband. Be satisfied with my friendship
for, I repeat, that is all I can promise you and I only promise what I
am sure of being able to give.'

`Yes, I understand,' said Fernand. `You bear your own poverty patiently,
but you are afraid of mine. Well, Merc\'ed\`es, with your love, I would try
to make my fortune; you would bring me luck and I should become rich. I
can cast my fisherman's net wider, I can take a job as a clerk in a
shop, I could even become a merchant myself!'

`You can't do any such thing, Fernand: you're a soldier and, if you stay
here among the Catalans, it is because there is no war for you to fight.
So remain a fisherman, don't dream of things that will make reality seem
even more terrible to you -- and be content with my friendship, because
I cannot give you anything else.'

`You are right, Merc\'ed\`es, I shall be a seaman; and, instead of the dress
of our forefathers which you despise, I shall have a patent-leather hat,
a striped shirt and a blue jacket with anchors on the buttons. That's
how a man needs to dress, isn't it, if he wants to please you?'

`What do you mean?' Merc\'ed\`es asked, with an imperious look. `What do you
mean? I don't understand you.'

`What I mean, Merc\'ed\`es, is that you are only so hard-hearted and cruel
towards me because you are waiting for someone who is dressed like that.
But it may be that the one you await is fickle and, even if he isn't,
the sea will be fickle for him.'

`Fernand!' Merc\'ed\`es exclaimed. `I thought you were kind, but I was
mistaken. It is wicked of you to call on the wrath of God to satisfy
your jealousy. Yes, I will not deny it: I am waiting for the man you
describe, I love him and if he does not return, instead of blaming the
fickleness that it pleases you to speak of, I shall think that he died
loving me.'

The young Catalan made an angry gesture.

`I understand what that means, Fernand: you want to blame him because I
do not love you, and cross his dagger with your Catalan knife! What good
would that do you? If you were defeated, you would lose my friendship;
if you were the victor, you would see that friendship turn to hatred.
Believe me, when a woman loves a man, you do not win her heart by
crossing swords with him. No, Fernand, don't be carried away by evil
thoughts. Since you cannot have me as your wife, be content to have me
as a friend and a sister. In any case,' she added, her eyes anxious and
filling with tears, `stay, Fernand: you said, yourself, a moment ago
that the sea is treacherous. It is already four months since he left,
and I have counted a lot of storms in the past four months!'

Fernand remained impassive. He made no attempt to wipe the tears that
were running down Merc\'ed\`es cheeks, yet he would have given a glass of
his own blood for each of those tears; but they were shed for another.
He got up, walked round the hut and returned, stopping before Merc\'ed\`es
with a dark look in his eyes and clenched fists.

`Come now, Merc\'ed\`es,' he said. `Answer me once more: have you truly made
up your mind?'

`I love Edmond Dant\`es,' the young woman said, coldly, `and no one will
be my husband except Edmond.'

`And you will love him for ever?'

`As long as I live.'

Fernand bent his head like a discouraged man, gave a sigh that was like
a groan, then suddenly looked up with clenched teeth and nostrils
flared.

`But suppose he is dead?'

`If he is dead, I shall die.'

`And if he forgets you?'

`Merc\'ed\`es!' cried a happy voice outside the house. `Merc\'ed\`es!'

`Ah!' the girl exclaimed, reddening with joy and leaping up, filled with
love. `You see that he has not forgotten me: he is here!' And she ran to
the door, and opened it, crying: `Come to me, Edmond! I am here!'

Pale and trembling, Fernand stepped back as a traveller might do at the
sight of a snake; and, stumbling against his chair, fell back into it.

Edmond and Merc\'ed\`es were in each other's arms. The hot Marseille sun,
shining through the doorway, drenched them in a flood of light. At
first, they saw nothing of what was around them. A vast wave of
happiness cut them off from the world and they spoke only those
half-formed words that are the outpourings of such intense joy that they
resemble the expression of pain.

Suddenly, Edmond noticed the sombre figure of Fernand, pale and
threatening in the darkness. With a gesture of which he was not even
himself aware, the young Catalan had laid his hand on the knife at his
belt.

`Oh, forgive me,' Dant\`es said, raising an eyebrow. `I did not realize
that we were not alone.'

Then, turning to Merc\'ed\`es, he asked: `Who is this gentleman?'

`He will be your best friend, Dant\`es, because he is my friend, my cousin
and my brother: this is Fernand, which means he is the man whom, after
you, I love most in the world. Don't you recognize him?'

`Ah! Yes, indeed,' said Edmond. And, without leaving Merc\'ed\`es whose hand
he held clasped in one of his own, he extended the other with a cordial
gesture towards the Catalan. But Fernand, instead of responding to this
sign of friendship, remained as silent and motionless as a statue. It
was enough to make Edmond look enquiringly from Merc\'ed\`es, who was
trembling with emotion, to Fernand, sombre and threatening.

That one glance told him everything. His brow clouded with rage.

`I did not realize that I had hurried round to see you, Merc\'ed\`es, only
to find an enemy here.'

`An enemy!' Merc\'ed\`es exclaimed, looking angrily in the direction of her
cousin. `An enemy, in my house, you say, Edmond! If I thought that, I
should take your arm and go with you to Marseille, leaving this house,
never to return.'

Fernand's eyes lit up with rage.

`And if any misfortune were to befall you, my dear Edmond,' she
continued, with the same cool determination, proving to Fernand that she
had read the sinister depths of his mind, `if any misfortune should
happen to you, I should climb up the Cap de Morgiou and throw myself
headlong on to the rocks.'

The blood drained from Fernand's face.

`But you are wrong, Edmond,' she continued. `You have no enemies here.
The only person here is Fernand, my brother, who is going to shake your
hand like a true friend.'

With these words, the girl turned her imperious face towards the Catalan
and he, as if mesmerized by her look, slowly came across to Edmond and
held out his hand. His hatred, like an impotent wave, had been broken
against the ascendancy that the woman exercised over him. But no sooner
had he touched Edmond's hand than he felt he had done all that it was
possible for him to do, and rushed out of the house.

`Ah!' he cried, running along like a madman and burying his hands in his
hair. `Ah! Who will deliver me from this man? Wretch that I am, wretch
that I am!'

`Hey, Catalan! Hey, Fernand! Where are you going?' a voice called to
him.

The young man stopped dead, looked around and saw Caderousse at the
table with Danglars under a leafy arbour.

`What now,' said Caderousse, `why don't you join us? Are you in such a
hurry that you don't have time to say hello to your friends?'

`Especially when they still have an almost full bottle in front of
them,' Danglars added.

Fernand stared at the two men with a dazed look, and did not answer.

`He seems a bit down in the dumps,' Danglars said, nudging Caderousse
with his knee. `Could we be wrong? Contrary to what we thought, could it
be that Dant\`es has got the upper hand?'

`Why! We'll just have to find out,' said Caderousse. And, turning back
to the young man, he said: `Well, Catalan, have you made up your mind?'

Fernand wiped the sweat from his brow and slowly made his way under the
vault of leaves: its shade appeared to do something to calm his spirits
and its coolness to bring a small measure of well-being back to his
exhausted body.

`Good day,' he said. `I think you called me?'

`I called you because you were running along like a madman and I was
afraid you would go and throw yourself into the sea,' Caderousse said
with a laugh. `Devil take it, when one has friends, it is not only to
offer them a glass of wine, but also to stop them drinking three or four
pints of water.'

Fernand gave a groan that resembled a sob and let his head fall on to
his wrists, which were crossed on the table.

`Well now, do you want me to tell you what, Fernand?' Caderousse
continued, coming straight to the point with that crude brutality of the
common man whose curiosity makes him forget any sense of tact. `You look
to me like a man who has been crossed in love!' He accompanied this quip
with a roar of laughter.

`Huh!' Danglars retorted. `A lad built like that is not likely to be
unhappy in love. You must be joking, Caderousse.'

`Not at all,' the other said. `Just listen to him sigh. Come, Fernand,
come now, lift your nose off the table and tell us: it is not very
mannerly to refuse to answer your friends when they are asking after
your health.'

`My health is fine,' said Fernand, clenching his fists and without
looking up.

`Ah, Danglars, you see now,' Caderousse said, winking at his friend.
`This is how things are: Fernand here, who is a fine, brave Catalan, one
of the best fishermen in Marseille, is in love with a beautiful girl
called Merc\'ed\`es; but it appears that, unfortunately, the girl herself is
in love with the second mate of the \emph{Pharaon}; and, as the
\emph{Pharaon} came into port this very day \ldots{} You follow me?'

`No, I don't,' said Danglars.

`Poor Fernand has got his marching orders,' Caderousse continued.

`So, what then?' said Fernand, lifting his head and looking at
Caderousse, like a man anxious to find someone on whom to vent his
wrath. `Merc\'ed\`es is her own woman, isn't she? She is free to love
whomsoever she wants.'

`Oh, if that's how you take it,' said Caderousse, `that's another
matter. I thought you were a Catalan, and I have been told that the
Catalans are not men to let themselves be pushed aside by a rival. They
even said that Fernand, in particular, was fearsome in his vengeance.'

Fernand smiled pityingly. `A lover is never fearsome,' he said.

`Poor boy!' Danglars continued, pretending to grieve for the young man
from the bottom of his heart. `What do you expect? He didn't imagine
that Dant\`es would suddenly return like this; he may have thought him
dead, or unfaithful. Who knows? Such things are all the more distressing
when they happen to us suddenly.'

`In any event,' Caderousse said, drinking as he spoke and starting to
show the effects of the heady wine of La Malgue, `in any event, Fernand
is not the only person to have been put out by Dant\`es' fortunate return,
is he, Danglars?'

`No, what you say is true -- and I might even add that it will bring him
misfortune.'

`No matter,' Caderousse went on, pouring out some wine for Fernand and
replenishing his own glass for the eighth or tenth time (though Danglars
had hardly touched the one in front of him). `No matter. In the meantime
he will marry Merc\'ed\`es, the lovely Merc\'ed\`es. He has come back for that,
at least.'

While the other was speaking, Danglars directed a piercing look at the
young man, on whose heart Caderousse's words were falling like molten
lead.

`And when is the wedding?' he asked.

`Oh, it's not settled yet,' Fernand muttered.

`No, but it will be,' said Caderousse, `just as surely as Dant\`es will be
captain of the \emph{Pharaon}, don't you think, Danglars?'

Danglars shuddered at this unexpected stab and turned towards
Caderousse, studying his face now to see if the blow had been
premeditated; but he saw nothing except covetousness on this face,
already almost besotted with drink.

`Very well,' he said, filling the glasses. `Then let's drink to Captain
Edmond Dant\`es, husband of the beautiful Catalan!'

Caderousse lifted his glass to his lips with a sluggish hand and drained
it in one gulp. Fernand took his and dashed it to the ground.

`Ha, ha!' said Caderousse. `What can I see over there, on the crest of
the hill, coming from the Catalan village? You look, Fernand, your
eyesight is better than mine. I think I'm starting to see less clearly
and, as you know, wine is a deceptive imp: it looks to me like two
lovers walking along, side by side and hand in hand. Heaven forgive me!
They don't realize that we can see them and, look at that, they're
kissing each other!'

Danglars marked every single trait of the anguish that crossed Fernand's
face, as its features changed before his eyes.

`Do you know who they are, Monsieur Fernand?' he asked.

`Yes,' the other replied dully. `It's Monsieur Edmond and Mademoiselle
Merc\'ed\`es.'

`There! You see?' said Caderousse. `I didn't recognize them. Hey,
Dant\`es! Hey, there, pretty girl! Come down for a moment and let us know
when the wedding is: Fernand here is so stubborn, he won't tell us.'

`Why don't you be quiet!' said Danglars, pretending to restrain
Caderousse who, with drunken obstinacy, was leaning out of the arbour.
`Try to stay upright and let the lovers enjoy themselves in peace. Why,
look at Monsieur Fernand: he's being sensible. Why not try and do the
same?'

It may be that Fernand, driven to the limit and baited by Danglars like
a bull by the banderilleros, would finally have leapt forward, for he
had already stood up and appeared to be gathering strength to throw
himself at his rival; but Merc\'ed\`es, upright and laughing, threw back her
lovely head and shot a glance from her clear eyes. At that moment,
Fernand recalled her threat to die if Edmond should die, and slumped
back, discouraged, on his chair.

Danglars looked at the two men, one besotted by drink, the other
enslaved by love, and murmured: `I shall get nothing out of these
idiots: I fear I am sitting between a drunkard and a coward. On the one
hand, I have a man eaten up by envy, drowning his sorrows in drink when
he should be intoxicated with venom; on the other, a great simpleton
whose mistress has just been snatched away from under his very nose, who
does nothing except weep like a child and feel sorry for himself. And
yet he has the blazing eyes of a Spaniard, a Sicilian or a Calabrian --
those people who are such experts when it comes to revenge -- and fists
that would crush a bull's head as surely as a butcher's mallet. Fate is
definitely on Edmond's side: he will marry the beautiful girl, become
captain and laugh in our faces. Unless \ldots' (a pallid smile hovered
on Danglars' lips) ` \ldots{} unless I take a hand in it.'

Caderousse, half standing, with his fists on the table, was still
shouting: `Hello, there! Hello! Edmond! Can't you see your friends, or
are you too proud to talk to them?'

`No, my dear Caderousse,' Edmond replied. `I am not proud, but I am
happy -- and happiness, I believe, is even more dazzling than pride.'

`At last, all is explained,' said Caderousse. `Ho! Good day to you,
Madame Dant\`es.'

Merc\'ed\`es bowed gravely and said: `That is not yet my name, and in my
country they say it is bad luck to call a young woman by the name of her
betrothed before he has become her husband. So, please, call me
Merc\'ed\`es.'

`You must forgive my good neighbour, Caderousse,' Dant\`es said. `He so
seldom makes a mistake!'

`So, the wedding is to take place shortly, Monsieur Dant\`es?' Danglars
said, greeting the two young people.

`As soon as possible, Monsieur Danglars. Today, everything is to be
agreed at my father's house and tomorrow or, at the latest, the day
after, we shall have the engagement dinner here at La R\'eserve. I hope
that my friends will join us: you, of course, are invited, Monsieur
Danglars, and you, too, Caderousse.'

`And Fernand?' Caderousse asked, with a coarse laugh. `Will Fernand be
there as well?'

`My wife's brother is my brother,' Edmond said, `and both Merc\'ed\`es and I
should regret it deeply if he were to be separated from us at such a
time.'

Fernand opened his mouth to reply, but his voice caught in his throat
and he could not utter a single word.

`The agreement today, the engagement tomorrow or the day after: by
George! You're in a great hurry, Captain.'

`Danglars,' Edmond said with a smile, `I shall say the same to you as
Merc\'ed\`es did a moment ago: don't give me a title that does not yet
belong to me, it could bring me ill luck.'

`My apologies,' Danglars replied. `I was merely saying that you seem in
a great hurry. After all, we have plenty of time: the \emph{Pharaon}
will not set sail for a good three months.'

`One always hurries towards happiness, Monsieur Danglars, because when
one has suffered much, one is at pains to believe in it. But I am not
impelled by mere selfishness. I have to go to Paris.'

`Ah, indeed! To Paris. And will this be your first visit, Dant\`es?'

`Yes.'

`You have business there?'

`Not of my own, but a final request that I must carry out for our poor
Captain Lecl\`ere. You understand, Danglars, the mission is sacred to me.
In any event, don't worry. I shall be gone only as long as it takes to
go there and return.'

`Yes, yes, I understand,' Danglars said aloud; then he added, under his
breath: `To Paris, no doubt to deliver the letter that the marshal gave
him. By heaven! That letter has given me an idea -- an excellent idea!
Ah, Dant\`es, my friend, your name is not yet Number One on the register
of the \emph{Pharaon}.'

Then, turning back to Edmond who was leaving, he shouted: `\emph{Bon
voyage!}'

`Thank you,' Edmond replied, turning around and giving a friendly wave.
Then the two lovers went on their way, calm and happy as two chosen
souls heading for paradise.

\backmatter
\end{document}


